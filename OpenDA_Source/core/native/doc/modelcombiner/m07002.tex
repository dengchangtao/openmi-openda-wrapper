\documentclass[a4paper,12pt]{article}
\newcommand{\styles}{../styles}
\newcommand{\figs}{.}
\usepackage[eng]{\styles/costa}

\title{The COSTA ModelCombiner}
\author{Erwin Loots and Nils van Velzen}
\memonum{CTA memo200701}
\date{\today}

\begin{document}
\memotitlepage

\begin{vtlogsheet}
\vtlogentry{1.0}{EL}{2007-07-11}{Initial version}{CvV}
\vtfilelocation{$<$COSTA\_DIR$>$/doc/modelcombiner}
\end{vtlogsheet}

\tableofcontents

%===============================================================================

\section{Introduction}

One of the building blocks that COSTA offers is the 
{\em COSTA-model}
component.  {\em COSTA-model} objects have an internal state, and a number of
{\em methods} to set, change or get this state.  In the COSTA project, a
number of such methods has been designed.  The implementation is left to
the user: existing code may be used to create the methods of the
COSTA-model component.  COSTA also offers a set of building blocks to help
setting up a COSTA-model component.  It is not necessary that all methods are
available for a COSTA-model component: a component lacking certain methods
simply cannot be used for certain tasks. 

The {\em COSTA ModelCombiner} is a tool which can be used to create a new
COSTA-model component. It combines two or more COSTA-model componentes into
one COSTA-model component. The combined method, including its methods, of
the combined model can be configured using an XML input file. 

The COSTA ModelCombiner is a generic tool for the construction of larger
COSTA-model componentes, and may be used to couple all kinds of COSTA models.
A specific type of combination for which the ModelCombiner is especially
intended, is the coupling of a deterministic (simulation) model and a noise
model into a stochastic model. 

Combining a deterministic model and a noise model is an important buidling
block in data assimilation, because existing simulation models are in
general deterministic, and many data assimilation methods need a stochastic
model.  This means that existing models have to be extended into a
stochastic model before assimilation techniques can be applied.

This memo gives a description of this generic tool called COSTA 
ModelCombiner.




\section{COSTA-models}\label{Sec: models}
\subsection{Mathematical description of a COSTA-model}
\label{Sec: model math}
The COSTA ModelCombiner is a tool for the construction of COSTA model
components. Its explanation requires a clear understanding of what a
COSTA-model is. This section is intended to explain the COSTA-model
component in sufficient detail. 

COSTA-models are COSTA components and therefore have an state (value)
and an interface. COSTA-models are intended to describe stochastic models,
which means that a model is available for the uncertainties 
(differences between model results and reality). Deterministic models 
are seen as a special case of a stochastic model, in which the uncertainties 
are ignored (assumed zero).

The 'value' of a COSTA-model $s=(x,u,g,G^u,W^u,G^A,W^A,t)$
consists of the following three parts:
\begin{itemize}
\item $x=(\phi,p^u,p^A)$: the 'extended state' (extended solution),
consisting of
     \begin{itemize}
        \item $\phi$: 'model state' or (model solution)
        \item $p^u$: forcing-noise parameters
        \item $p^A$: operator-noise parameters
     \end{itemize}
\item $u$: the {\em forcings}
\item $g$: the {\em parameters} or {\em schematization} of the model.
\item $W^u$, $W^A$, $G^u$, $G^A$: interpolation and covariance matrices used 
      in noise models.
\item $t$ internal 'time' of the model (not very important in this memo).
\end{itemize}

In the case of a deterministic simulation, there are no noise parameters. 
The model state is {\em propagated} in the following way.
\begin{eqnarray}\label{Eq:GeneralModel}
\phi\left( t_{i+1} \right) &=& 
 A \left[ \phi \left(t_i\right), u \left( t_i \right), g\right]:
\end{eqnarray}
The new values are calculated from the old values using some (often very
complicated) function $A$, under the influence of
the current values for the forcings $u$ and the values of the parameters
$g$, which are time-independent.


In the general case of a stochastic model, uncertainties are included
in the propagation equation. The propagation of the 'true' state $\phi^t$ 
is assumed to follow the propagation operator $A$, except for an error 
$\delta A$:
\begin{eqnarray}
\phi^t \left(t_{i+1} \right) &=& A \left[ \phi^t \left( t_i \right),
                                 u^t \left(t_i \right), 
                                 g \right] + \delta A \left( t_i \right).
\end{eqnarray}
The 'true' state $\phi^t$ is of only found when the 'true' previous state, 
forcings and parameters are entered into the propagation operator. These
'true' values are thought to be given by the forecast values (indicated 
with the superscript $f$) and an error term:

\begin{eqnarray}
\phi^t(t_i) &=& \phi^f \left(t_i \right) +
                                 \delta \phi\left(t_i\right),
\nonumber \\
u^t\left(t_i\right) &=& u^f\left(t_i\right) +
                        \delta u \left(t_i\right),
\nonumber \\
g^t &=& g^f+ \delta g
\label{Eq: error terms in phi, u, g}
\end{eqnarray}
The error terms $\delta A$ and $\delta u$ can be
obtained (through interplation) from the {\em noise parameter} vectors 
$p^A$ and $p^u$ with a much smaller dimension:
\begin{eqnarray}
 \delta u(t_i) = W^u p^u(t_i) &,&
 \delta A(t_i) = W^A p^A(t_i).
\end{eqnarray}
The noise parameters may be described by AR(1) processes:
\begin{eqnarray}
   \delta u^f(t_{i+1}) &=& \mbox{diag}(\alpha^u) \delta u(t_i) +
\eta^u(t_i),
\nonumber \\
   \delta A^f(t_{i+1}) &=& \mbox{diag}(\alpha^A) \delta A(t_i) + 
\eta^A(t_i),
\label{eq: AR 1}
\end{eqnarray}
Where $\eta^A$ and $\eta^u$ are normally distributed stochastic variables, 
with their covariance matrices $G^A$ and $G^u$ given by 
\begin{eqnarray}
G^u = E(\eta^u(t_i) \eta^u(t_i)^T) &,& 
G^A = E(\eta^A(t_i) \eta^A(t_i)^T).
\end{eqnarray}

The propagation of the complete system is described in the following
equation:
\begin{eqnarray}
 x^t(t_{i+1}) = A_x[x^t(t_i),u(t_i),g] + \left( \begin{array}{l} 0 \\ \eta
  \label{eq: alles-vergelijking}
\end{array}\right),
\end{eqnarray}
where 
\begin{itemize}
 \item The propagation operator $A_x$ is given by
  \begin{eqnarray}
      A_x[(\phi,p^u,p^A),u,g] = \left(
       \begin{array}{ll}
         A[\phi,u+W^up^u,g] + W^A p^A\\
         \mbox{diag}(\alpha^u)p^u\\
         \mbox{diag}(\alpha^A)p^A
      \end{array} \right)
  \end{eqnarray}
 \item The covariance matrix $G$ of the normally distributed vector $\eta$
       is given by
       \begin{eqnarray}
         G = \left(\begin{array}{ll}
                G^u & 0 \\ 0 & G^A \end{array}
             \right).
       \end{eqnarray}
\end{itemize}
The internal structure of the model, given in equations (\ref{eq: AR 1}), 
is not essential for most data assimilation methods. 
The extended state vector $x$ may also be composed of different parts, and 
the extended propagation operator $A_x$ may have a different structure.
The overall
propagation equation (\ref{eq: alles-vergelijking}), however, is crucial,
because it is the starting point of most data-assimilation methods.

The interface of the COSTA model component was designed to perform all the
necessary manipulations of stochastic models of the kind described in this
section.  Models with a simpler structure are obtained by leaving certain
parts of the model empty. 



\subsection{Interface functions of the COSTA model component}
\label{Sec: model interface}
The previous section discussed the structure of COSTA models. The section
concluded by stating that the interface of the COSTA model component 
contains all the functions necessary to support data assimilation methods.

The COSTA ModelCombiner is a {\em COSTA component class}. 
This means that it is one of the possible {\em implementations} of the
COSTA model interface. Since this memo intends to describe the usage of 
the COSTA ModelBuilder and the way it works, it is important to know
all the functions in the interface.

A COSTA model provides the following functions:
\begin{itemize}
 \item {\tt DefineCLass}, {\tt Create}, {\tt Free}: 
                      functions necessary for the construction and
                      descruction of COSTA models.
 \item {\tt Compute}: carry out the time steps necessary to step through a 
                      given time span.

 \begin{itemize}
   \item FOR $i=i_{start},\cdots,i_{end}$, DO
    \begin{eqnarray}
       \phi &:=& A[\phi,u(t_i)+W^u p^u,g+\delta g] + W^A p^A,
    \nonumber \\
      p^u &:= & \mbox{diag}(\alpha^u)p^u + \eta^u(t_i),
    \nonumber \\
      p^A &:= & \mbox{diag}(\alpha^A)p^A + \eta^A(t_i).
    \end{eqnarray}
    END
 \end{itemize}

 \item {\tt AddNoise}: specify the noise which is to be added to the
                      forcings, state and schematization;
 \begin{itemize}
   \item FOR $i=i_{start},\cdots,i_{end}$, DO
    \begin{eqnarray}
      \eta^u(t_i) &:= & G^u~ {\tt randn(size(}p^u{\tt ))}
    \nonumber \\
      \eta^A(t_i) &:= & G^A~ {\tt randn(size(}p^A{\tt ))}
    \end{eqnarray}
    END
 \end{itemize}

 \item {\tt SetState}, {\tt GetState}, {\tt Axpy}: 
                      set, return or modify the state values $\phi$
                      in the form of a {\em COSTA state vector} object;
 \item {\tt SetForc}, {\tt GetForc}, {\tt AxpyForc}: 
                      set, return or modify the forcing values $u$
                      in the form of a {\em COSTA state vector} object;
 \item {\tt SetParam}, {\tt GetParam}, {\tt AxpyParam}: 
                      set, return or modify the model schemetization $g$
                      in the form of a {\em COSTA state vector} object; 
\item {\tt GetNoiseCount}, {\tt GetNoiseCovar}: 
                      return (dimension of) covariance matrix $G$ of the 
                      noise model;
 \item {\tt GetObsValues}: interpolate the model state to the observations.
 \item {\tt GetObsSelect}: return information which can be used to
                      read only the observations from a 
                      {\em COSTA stochastic observer} object for which
                      predictions can be generated by the model.
\end{itemize}

Using the existing COSTA models, the ModelCombiner provides all the 
functions of the interface.


\subsection{Extension of the interface, necessary for the ModelCombiner}
In the current project, the interface of the COSTA model will be extended:
when getting, setting or updating the state, forcings or schematization,
so-called {\em meta-information} will be supplied to describe the
information given or asked. This will make it possible for the model to
interpret and meaningfully process the information given, or to supply the
correct information. It will also make it possible to get, set or change
only {\em part} of the state, forcings or schematization, because only the
information described by the meta-information is returned, set or changed.
The new interface makes it (much) easier to combine COSTA models,
beacuse it makes it possible to pass the information from one model to
another. 

The meta-information will have to be obtained from the COSTA models. 
This is similar to the meta-information which is given about the COSTA
Stochastic Observer component by the COSTA Observer Description component.  
The meta-information object is be constructed in an analogous way.

More specific, a metainfo structure is attached to each state. In general, it
consists of some detailed information, description and unit of the quantity. An
important part is the grid information. The vector in a state leaf lives on a
grid. Up to now only linear grids are supported.

The advantage is that in the case of an axpy-operation, the grid can be used to
perform necessary operations. In this way a coarse noise vector can be added to
a dense state vector.


Metainfo is not obligatory. States can be used without metainfo, as always was
the case, but operations are restricted somewhat. For example, an axpy
operation between two states is only possible if the states and their substates
have the same length.


{\bf Description of the current metainfo structure:}
\begin{itemize}
\item tag: a short unique description for identification purposes like 'sep' or
  'u'. This is used for binary operations like axpy: the function, when called
  with two large states as input, knowns that it performs the operation on
  substates with the same tag. Substates which do not have a matching
  metainfo-tag in any of the substates of the other state, are ignored.
  it can 
\item belongs\_to: a second tag, also for binary operations. The user can in
  this way specify that 'sep' and 'noise\_of\_sep' belong to each other.
\item unit: the unit of the quantity described.
\item grid This is a separate structure attached on the metainfo.
   \begin{itemize}
     \item type: 2D or 3D
     \item name
     \item x\_origin, y\_origin
     \item nx,ny,nz: dimension in each direction
     \item dx,dy,dz: distance between two grid points in each direction
   \end{itemize}


\end{itemize}

{\bf description of metainfo operations:}

CTA\-Metainfo\_Create(CTA\_Metainfo *minfo);

CTA\-Metainfo\_Copy(CTA\_Metainfo minfo1, CTA\_Metainfo minfo2 );

CTA\-Metainfo\_SetUnit(CTA\_Metainfo minfo, char* unit );

CTA\-Metainfo\_GetUnit(CTA\_Metainfo minfo, char* unit );

CTA\-Metainfo\_SetTag(CTA\_Metainfo minfo, char* tag );

CTA\-Metainfo\_GetTag(CTA\_Metainfo minfo, char* tag );

CTA\-Metainfo\_SetBelongsTo(CTA\_Metainfo minfo, char* tag );

CTA\-Metainfo\_GetBelongsTo(CTA\_Metainfo minfo, char* tag );

CTA\_Metainfo\_GetGrid(CTA\_Metainfo minfo, CTAI\_Gridm *hgrid);

CTA\_Metainfo\_SetGrid(CTA\_Metainfo minfo, CTAI\_Gridm *hgrid);



 


\section{Combining COSTA models}
\subsection{Combining noise models and a deterministic model}
\label{Sec: example combi}
A very important example of a combined model is the combination of a 
deterministic model and a noise model. 
\begin{itemize}
\item The {\bf deterministic model} has the state $s_d = (\phi,u,g)$. There are
      no noise parameters.
      The propagation rule is:
\begin{eqnarray}
      \phi(t_{i+1}) := A[\phi(t_i),u(t_i),g]
\end{eqnarray}
\item The {\bf noise model for the forcings} has the state $s_u = (p^u)$. 
      The propagation rule is:
\begin{eqnarray} 
   p^u &:=& \mbox{diag}(\alpha^u) ~ p^u + \eta^u(t_i),
\nonumber \\
   \eta^u(t_i) &\equiv& N(0,1).
\end{eqnarray}
\item The {\bf noise model for the solution} has the state $s_A = (p^A)$. 
      The propagation rule is:
\begin{eqnarray} 
   p^A &:=& \mbox{diag}(\alpha^A) ~ p^u + \eta^A(t_i).
\nonumber \\
   \eta^A(t_i) &\equiv& N(0,1).
\end{eqnarray}
\end{itemize}
These three submodels are each considerably simpler than the stochastic
model described in Section \ref{Sec: model math}.

The state $\phi$ of the combined model is the concatenation of the 
states of the three submodels; so are the other . The COSTA state vector 
component has the functionality to handle concatenated vectors.

The same thing can be said for the forcings, schematization and 
noise parameters. The combined covariance matrix has a block diagonal
structure.

The propagation of the combined model consists of the following steps:
\begin{enumerate}
 \item Interpolate the forcings-parameters and add the forcings-noise to
       the forcings
    \begin{eqnarray}
       u(t_i) := u(t_i) + W^u p^u
    \end{eqnarray}
 \item Propagate the deterministic model
    \begin{eqnarray}
       \phi := A[\phi,u(t_i),g]
    \end{eqnarray}
 \item Interpolate the model-noise and add the model-noise to the solution 
    \begin{eqnarray}
       \phi := \phi + W^A p^A. 
    \end{eqnarray}
 \item Propagate forcings-noise
    \begin{eqnarray}
       p^u := \mbox{diag}(\alpha^u)~p^u + \eta_u(t_i)
    \end{eqnarray}
 \item Propagate model-noise
    \begin{eqnarray}
       p^A := \mbox{diag}(\alpha^A)~p^A + \eta_A(t_i)
    \end{eqnarray}
\end{enumerate}
In this example, it is clear how the submodels should be combined into a 
combined model, using the functions in the interface of the COSTA model
component, and interpolations needed for the commnunication between the
submodels.


\subsection{Configuration file for ModelCombiner}
\label{Sec: usage}
The previous sections provide some insight into the way a combined 
model will work and the things that must be specified to the ModelCombiner.
In this Section, it will be explained how the COSTA user (i.e. the model 
programmer) can describe the coupled model to the ModelCombiner.

The information is presented to the ModelCombiner in the form of a 
configuration file, written in XML.

The overall structure of an input file for the ModelCombiner is given
in Table \ref{Tab: example input file}. It consists of two parts: the 
definitions of the submodels and the specification of the propagation step 
of the combined model.


\begin{table}
\begin{tabular}{|l|}
\hline
{\small
\begin{minipage}{15cm}
\begin{verbatim}

  <modelbuilder model="stochastic model" tagstate="a" 
                      tagforc="b" tagparam="c">


   <submodels>
\end{verbatim}
\hspace{1.4cm}
{\em submodel definitions, between {\tt <submodel>}}
\begin{verbatim}
   </submodels>

   <propagation>
       <deterministic>
\end{verbatim}
\hspace{1.2cm}
\begin{tabular}{p{12cm}}
\begin{minipage}{12cm}
\begin{itemize}
\item[]
{\em things to be done for the propagation of the combined, extended
     solution, between {\tt <action type=*>}, except the propagation of 
     stochastic models}
\end{itemize}
\end{minipage}
\end{tabular}
\begin{verbatim}
       </deterministic>
       <stochastic>
          <action type=propagate> boundary noise model  </action>
          <action type=propagate> wind noise model      </action>
          <action type=propagate> viscosity noise model </action>
          <action type=propagate> velocity noise model  </action>
       </stochastic>
   </propagation>

</modelbuilder>

\end{verbatim}
\end{minipage}
}
\\ \hline
\end{tabular}
\caption{\em Overall structure of the input file for the ModelCombiner}
\label{Tab: example input file}
\end{table}

An example of the definitions of the submodels is given in Table \ref{Tab:
submodel definitions}. Every submodel is given a name. Some additional 
information is necessary since the combined model creates its own submodels. 

\begin{table}
\begin{tabular}{|l|}
\hline
{\small
\begin{minipage}{15cm}
\begin{verbatim}

<submodel> 
       <name>         deterministic model      </name>
       <model_class>  CTA_WAQUA_MODEL          </model_class>
       <create_input> control_simona.txt       </create_input>
</submodel>

<submodel>
       <name>         boundary noise model     </name>
       <model_class>  CTA_MODEL_BUILDER        </model_class>
       <create_input> boundary_noise_model.xml </create_input>
</submodel>
\end{verbatim}
{\em submodel definitions for {\tt wind noise model}, 
{\tt viscosity noise model} and {\tt velocity noise model}, similar to that 
of {\tt boundary noise model}}\\[1ex]
\end{minipage}}\\
\hline
\end{tabular}
\caption{Example of the submodel definitions in the input file}
\label{Tab: submodel definitions}
\end{table}

Table \ref{Tab: example deterministic propagation} gives an example of the
'actions' which constitute the propagation of the extended solution in the
combined model. The example is very similar to (but a little more extended
than) the steps given in Section \ref{Sec: example combi}.  Every step in
the propagation is called an 'action'.  Several kinds of actions are
distiguished:
\begin{itemize}

 \item {\tt set}, {\tt get}, {\tt axpy}:

       Set , get or adjust a part of the submodel state, forcings or
       parameters, using the values and meta-information of the state,
       forcings or parameters of an other submodel.

 \item {\tt propagate}:
 
       Carry out time step calculation(s) until 
       the desired simulated time level.

\end{itemize}

\begin{table}
\begin{tabular}{|l|}
\hline
{\small
\begin{minipage}{15cm}
\begin{verbatim}

<action type="x=axpy"> 
      <var_y  var="state">    boundary noise model </var_y>
      <const_a>               1.0                  </const_a>
      <var_x var="forcings"> deterministic model  </var_x>
</action>

\end{verbatim}
{\em forcing adaptations for {\tt wind noise model} and
{\tt viscosity noise model}, similar to that 
of {\tt boundary noise model}}\\
\begin{verbatim}
<action type="propagate">    deterministic model     </action>

<action type="y=axpy"> 
      <var_x  var="state"> velocity noise model    </var_x>
      <const_a>            1.0                     </const_a>
      <var_y var="state"> deterministic model     </var_y>
</action>

\end{verbatim}
\end{minipage}
}
\\ \hline
\end{tabular}
\caption{\em Example of the specification of all the steps in the 
             propagation of the combined model, except propagation of
             stochastic models.}
\label{Tab: example deterministic propagation}
\end{table}




%--------------------------

\subsection{The AR(1) noise model}

Most users start with their own deterministic model. To make things easier,
this model can be combined with a standard AR(1) noise model provided by COSTA.
The AR(1) model can be created using a special variant of the Model Builder.
That means that the functions that normally have to be provided by the user,
are now already given by COSTA. The only aspects that the user has to specify
are the characteristic parameters and the grid. An example of the specification
in XML is shown in Table \ref{Tab: ar1}. The XML-code represents the 
{\verb <create_input>} entry in the submodel definition.


\begin{table}
\begin{tabular}{|l|}
\hline
{\small
\begin{minipage}{15cm}

\begin{verbatim}

       <modelbuild_sp>
          <special_ar1>
            <tag>noise-bc-n</tag>
            <parameters std_dev="1.0" kar_t="105.90" kar_l="0.5">
            </parameters>
            <grid>
               <type_id>2D</type_id>
               <gridsize  nx="4"  ny="1">  </gridsize>
               <gridparams x_origin="0.0" y_origin="2.5"
                          dx="0.5" dy="0.0">
               </gridparams>
            </grid>
          </special_ar1>

       </modelbuild_sp>




\end{verbatim}



{\em submodel definition: an AR(1) noise model} \\[1ex]
\end{minipage}}\\
\hline
\end{tabular}
\caption{Example of the AR(1) submodel in the input file}
\label{Tab: ar1}
\end{table}

The necessary parameters are the standard deviation {\tt std\_dev}, the
characteristic time {\tt kar\_t} and the characteristic length {\tt kar\_l}.
$$
x_{n+1} = \alpha \cdot x_{n} + \sqrt{1 - \alpha^2} \cdot\eta
$$
where $\alpha = \exp{(- \delta t  kar_t)} $ and $\eta$ the covariance matrix
  computed from the grid and the characteristic length {\tt kar\_l}.


\section{examples}
\subsection{oscillation model}
As a first example, the oscillation model (without noise) has been 
combined with an AR(1) process. Note that the grid specification is in this
case not entirely logical since the first element of the state describes the
(non-constant) position. However, this example will provide some insight in the
working of the ModelCombiner.

Files:

{\tt models\slash oscill\slash oscill.f90}

{\tt models\slash oscill\slash c\_oscill.c}

{\tt applications\slash da\_tools\slash ens\_proto\slash ens\_combine\_oscill\_ar1.f90}

{\tt applications\slash da\_tools\slash ens\_proto\slash ens\_combine\_oscill\_ar1.xml}

{\tt applications\slash da\_tools\slash ens\_proto\slash sm\_oscill\_model.xml}

{\tt applications\slash da\_tools\slash ens\_proto\slash sm\_ar1\_model.xml}


\subsection{heat model}
As a second example, a second instance of the heat model has been made.
The model consists now of the single temperature state, and five forcings:
the heat and four boundary  condition states. The idea is that the forcings are
adjusted using the axpy operation with the corresponding five ar(1) models. In
this case, the grids of the ar(1) models are twice as coarse as the grids where
the forcings live.

Files:

{\tt models\slash heat\_modelcombiner\slash heat\_model.f}

{\tt models\slash heat\_modelcombiner\slash c\_heatmodel.c}

{\tt applications\slash da\_tools\slash ens\_proto\slash ens\_combine\_heat\_ar1.f90}

{\tt applications\slash da\_tools\slash ens\_proto\slash ens\_combine\_heat\_ar1.xml}

{\tt applications\slash da\_tools\slash ens\_proto\slash sm\_ar1\_model\_nheat\_n.xml}

{\tt applications\slash da\_tools\slash ens\_proto\slash sm\_ar1\_model\_nbc\_n.xml}

{\tt applications\slash da\_tools\slash ens\_proto\slash sm\_ar1\_model\_nbc\_e.xml}

{\tt applications\slash da\_tools\slash ens\_proto\slash sm\_ar1\_model\_nbc\_w.xml}

{\tt applications\slash da\_tools\slash ens\_proto\slash sm\_ar1\_model\_nbc\_s.xml}


\end{document}
